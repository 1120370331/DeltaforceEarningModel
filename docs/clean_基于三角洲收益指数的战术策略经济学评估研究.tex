\documentclass[12pt]{ctexart}

\usepackage[a4paper,margin=2.5cm]{geometry}
\usepackage{amsmath,amssymb}
\usepackage{booktabs}
\usepackage{graphicx}
\usepackage{hyperref}

\title{基于三角洲收益指数的战术策略经济学评估研究\\[4pt]
(含情景化与参数敏感性分析)}
\author{}
\date{}

\begin{document}
\maketitle

\begin{abstract}
本文围绕一类高风险高收益的局内战术玩法,构建统一的三角洲收益指数模型,用于度量不同玩法在综合考虑收益、损耗、成本、时间与成功率后的经济效率。在原始模型基础上,引入每分钟至少 1 万哈夫币的时间机会成本,将时间显式货币化,得到修正后的收益强度指标 $RI^{*}$。

基于八种代表性策略(全装猛攻、玻璃大炮猛攻、四套半装猛攻机密、绝密避战跑刀、经济学绝密跑刀、机密大坝速撤、普通经济学、全装单三猛攻)给定的经验参数,本文计算各策略的 $RI^{*}$,以及作为对照的两类辅助指标 $J_1,J_2$,并进行基准排序。随后,构造两个情景:其一是固排配合与技术意识提升,使所有“猛攻”策略胜率提高 20\%;其二是账号强度较低但爆率较高,使所有“绝密跑刀”策略的收益提高 50\%,撤离率提高 10\%。

在此基础上,本文比较三种候选指标形式 $J_1,J_2,RI^{*}$ 的含义与优缺点,论证选择 $RI^{*}$ 作为主指标的理由;并将时间成本单价 $\kappa$ 视为可调参数,对 $RI^{*}$ 的参数敏感性进行分析。进一步从期望收益拓展到“破产风险”和“爽度--经济效率”双目标框架,对后续研究方向进行展望。

主要结论包括:普通经济学与机密大坝速撤在经济意义上始终具备显著优势;在收益上调后,绝密避战跑刀与经济学绝密跑刀的经济效率明显提升;各类猛攻策略即便在胜率改善的情景下,经济效率仍明显劣于跑刀与速撤策略。新增的全装单三猛攻在经济性上介于玻璃大炮猛攻与四套半装猛攻机密之间。数值分析显示,在较宽的时间价值区间内,前两名始终由普通经济学与机密大坝速撤占据,仅在较大的 $\kappa$ 下发生名次互换。
\end{abstract}

\section{模型构建}

\subsection{变量与单位}

统一约定:
\begin{itemize}
    \item 金额单位:万哈夫币;
    \item 时间单位:分钟。
\end{itemize}

对每一局行为(一次玩法)定义变量如下:
\begin{itemize}
    \item 单局平均营收额:$AR$(万哈夫币),成功结算时的期望总收入(不含成本扣除);
    \item 单局平均损耗额:$AL$(万哈夫币),因装备折损、资源消耗等产生的货币化损失;
    \item 显性入局成本:$C$(万哈夫币),开局所需的固定费用或装备投入;
    \item 撤离成功率:$SR\in[0,1]$,单局成功撤离或顺利结算的概率;
    \item 单局平均耗时:$T$(分钟),完成或终止一局所需的平均时间。
\end{itemize}

定义:
\begin{equation}
\Delta R = AR - AL,\qquad
\Delta R_{\text{adj}} = (AR - AL)\times SR,
\end{equation}
分别代表净收益与风险调整后净收益(考虑成功率后的期望净收益)。

\subsection{原始三角洲收益指数(不计时间成本)}

不考虑时间机会成本时,可以定义原始三角洲收益指数:
\begin{equation}
RI = \frac{(AR - AL)\times SR}{C \times T}
= \frac{\Delta R_{\text{adj}}}{C\,T}.
\end{equation}

含义是:在显性成本 $C$ 与耗时 $T$ 的约束下,单位“显性成本$\times$时间”投入所获得的风险调整后期望净收益强度。

此形式存在两个问题:
\begin{enumerate}
    \item 对于 $C=0$ 且 $\Delta R_{\text{adj}}>0$ 的名义零成本玩法(如普通经济学),分母为 0,$RI$ 无法定义;
    \item 时间本身未被赋予“机会成本”,不能体现“即便不花钱,也在牺牲可以做别的事的时间”的现实。
\end{enumerate}

\subsection{加入时间成本的修正模型}

为刻画“时间也是成本”,引入统一的时间机会成本单价:
\begin{equation}
\kappa = 1 \quad (\text{万哈夫币/分钟}),
\end{equation}
即每使用 1 分钟时间,至少承担 $\kappa$ 万哈夫币的机会成本。

则一局行为的时间机会成本为:
\begin{equation}
\mathrm{TimeCost} = \kappa T.
\end{equation}

定义有效总成本:
\begin{equation}
C_{\text{eff}} = C + \kappa T,
\end{equation}
即:总成本 = 显性成本 + 时间机会成本。

据此得到修正后的三角洲收益指数:
\begin{equation}
\boxed{
RI^{*}
= \frac{(AR - AL)\times SR}{(C + \kappa T)\,T}
= \frac{\Delta R_{\text{adj}}}{C_{\text{eff}}\,T}
}
\end{equation}

性质与直观含义如下:
\begin{itemize}
    \item 即便 $C=0$,只要 $T>0$,就有 $C_{\text{eff}}>0$,避免了除以 0;
    \item 长时间局由于同时增大 $C_{\text{eff}}$ 与 $T$,在经济性上受到双重惩罚,体现了“又贵又耗时”的策略在纯经济视角下不被偏好;
    \item 指标单位为 $1/\text{分钟}$,可解释为在考虑成本后的单位时间收益强度。
\end{itemize}

\subsection{指标形式对比与选择理由}

在构建指标时,除了 $RI^{*}$,还可以考虑以下两种更直观的形式:
\begin{enumerate}
    \item 只按时间归一化的指标:
    \begin{equation}
    J_1 = \frac{\Delta R_{\text{adj}}}{T},
    \end{equation}
    表示每分钟的期望净收益。
    \item 只按成本归一化的指标:
    \begin{equation}
    J_2 = \frac{\Delta R_{\text{adj}}}{C_{\text{eff}}},
    \end{equation}
    表示每一单位有效成本能“榨”出多少期望净收益。
\end{enumerate}

三者关系为:
\begin{equation}
RI^{*}=\frac{\Delta R_{\text{adj}}}{C_{\text{eff}}T}
=\frac{J_1}{C_{\text{eff}}}=\frac{J_2}{T}.
\end{equation}

对比可见:
\begin{itemize}
    \item $J_1$ 强调“赚钱速度”,对时间敏感,对成本相对宽容;
    \item $J_2$ 强调“资本效率”,对成本敏感,对时间相对宽容;
    \item $RI^{*}$ 同时对成本和时间都敏感,相当于对长时间、高成本策略施加“双重惩罚”。
\end{itemize}

本文选择 $RI^{*}$ 作为主指标,是基于如下考虑:
\begin{enumerate}
    \item 研究对象为高风险、高投入战术玩法,既昂贵又耗时,同时惩罚成本与时间更符合“纯经济学视角”的理性偏好;
    \item $RI^{*}$ 在维度上清晰,且易于用于情景化比较与参数敏感性分析;
    \item 对实际使用者而言,若偏好“时间优先”或“资本优先”,仍可辅以 $J_1,J_2$ 作为辅助排序指标。
\end{enumerate}

\section{策略集合与基准参数}

\subsection{策略定义}

本文考虑八种典型玩法:
\begin{enumerate}
    \item 全装猛攻;
    \item 玻璃大炮猛攻;
    \item 四套半装猛攻机密;
    \item 绝密避战跑刀(收益修订为 160 万);
    \item 经济学绝密跑刀;
    \item 机密大坝速撤;
    \item 普通经济学(时间修订为 2.5 分钟,收益修订为 4 万);
    \item 全装单三猛攻(新增:单人全装三人位装备水准)。
\end{enumerate}

\subsection{基准参数设定}

所有金额单位为万哈夫币,时间单位为分钟。表~\ref{tab:params} 给出了各策略的基准参数设定。

\begin{table}[htbp]
\centering
\caption{策略基准参数设定(单位:万哈夫币、分钟)}
\label{tab:params}
\begin{tabular}{lccccc}
\toprule
策略名称 & $C$ & $AL$ & $AR$ & $SR$ & $T$ \\
\midrule
全装猛攻         & 250 & 200 & 400  & $1/3$ & 25 \\
玻璃大炮猛攻     & 120 & 50  & 200  & 0.5   & 25 \\
四套半装猛攻机密 & 200 & 100 & 300  & 0.4   & 25 \\
绝密避战跑刀     & 70  & 0   & 160  & 0.5   & 15 \\
经济学绝密跑刀   & 50  & 0   & 70   & 0.2   & 5  \\
机密大坝速撤     & 15  & 0   & 35   & 0.7   & 5  \\
普通经济学       & 0   & 0   & 4    & 1.0   & 2.5 \\
全装单三猛攻     & 500 & 400 & 1100 & $1/3$ & 25 \\
\bottomrule
\end{tabular}

\vspace{4pt}
\footnotesize $C,AL,AR$ 单位均为万哈夫币,$T$ 单位为分钟,$SR$ 为无量纲概率。
\end{table}

\section{计算方法与基准排名}

\subsection{计算步骤}

对每个策略 $s$:
\begin{align}
\Delta R_s &= AR_s - AL_s,\\
\Delta R_{\text{adj},s} &= \Delta R_s \times SR_s,\\
C_{\text{eff},s} &= C_s + \kappa T_s,\quad \kappa = 1,\\
RI_s^{*} &= \frac{\Delta R_{\text{adj},s}}{C_{\text{eff},s}\,T_s},\\
J_{1,s} &= \frac{\Delta R_{\text{adj},s}}{T_s},\\
J_{2,s} &= \frac{\Delta R_{\text{adj},s}}{C_{\text{eff},s}}.
\end{align}

\subsection{基准情形下的指标结果($\kappa=1$)}

在基准时间成本单价 $\kappa = 1$ 万/分钟下,对每个策略计算 $RI^{*},J_1,J_2$。表~\ref{tab:indicators} 按 $RI^{*}$ 从高到低给出结果(四舍五入保留三位小数)。

\begin{table}[htbp]
\centering
\caption{基准情形下各策略的三类指标($\kappa=1$)}
\label{tab:indicators}
\begin{tabular}{lcccc}
\toprule
排名 & 策略 & $RI^{*}$ & $J_1$(每分钟净收益) & $J_2$(每单位成本净收益) \\
\midrule
1 & 普通经济学       & 0.640 & 1.600 & 1.600 \\
2 & 机密大坝速撤     & 0.245 & 4.900 & 1.225 \\
3 & 绝密避战跑刀     & 0.063 & 5.333 & 0.941 \\
4 & 经济学绝密跑刀   & 0.051 & 2.800 & 0.255 \\
5 & 玻璃大炮猛攻     & 0.021 & 3.000 & 0.517 \\
6 & 全装单三猛攻     & 0.018 & 9.333 & 0.444 \\
7 & 四套半装猛攻机密 & 0.014 & 3.200 & 0.356 \\
8 & 全装猛攻         & 0.010 & 2.667 & 0.242 \\
\bottomrule
\end{tabular}
\end{table}

从表~\ref{tab:indicators} 可见:
\begin{itemize}
    \item 在主指标 $RI^{*}$ 下,普通经济学与机密大坝速撤牢牢占据前两名,其次为两种绝密跑刀,最后是四种猛攻策略;
    \item 若只看赚钱速度 $J_1$,全装单三猛攻和绝密避战跑刀表现突出,说明其单局爆发很强,但在 $RI^{*}$ 下被成本与时间双重惩罚;
    \item 若只看资本效率 $J_2$,普通经济学与大坝速撤表现最优,绝密避战跑刀紧随其后。
\end{itemize}

为便于叙述,将 $RI^{*}$ 粗略分档:
\begin{itemize}
    \item S 级(高效率):$RI^{*} \ge 0.20$;
    \item A 级(中高效率):$0.05 \le RI^{*} < 0.20$;
    \item B 级(低效率):$RI^{*} < 0.05$。
\end{itemize}

据此:
\begin{itemize}
    \item S 级经济学效率流:普通经济学(0.640)、机密大坝速撤(0.245);
    \item A 级高效跑刀流:绝密避战跑刀(0.063)、经济学绝密跑刀(0.051);
    \item B 级高消耗猛攻流:玻璃大炮猛攻(0.021)、全装单三猛攻(0.018)、四套半装猛攻机密(0.014)、全装猛攻(0.010)。
\end{itemize}

以全装单三猛攻为例:
\begin{equation}
C_{\text{eff}} = 500 + 25 = 525,\qquad
\Delta R_{\text{adj}} = (1100 - 400)\times \frac{1}{3} \approx 233.33,
\end{equation}
从而
\begin{equation}
RI^{*} \approx \frac{233.33}{525\times 25} \approx 0.0178,
\end{equation}
明显优于全装猛攻,也略优于四套半装猛攻机密,但仍落后于玻璃大炮猛攻以及所有跑刀和速撤策略。

\section{情景化调整与策略排名变化}

\subsection{情景设定与建模假设}

\subsubsection{情景 A:固排配合与技术意识提升}

情景叙述:固排队友配合默契、技术意识明显提高,使得所有“猛攻”类玩法的胜率提高 20\%。

建模假设(相对提升):
\begin{equation}
SR' = \min(1,\,1.2\times SR),
\end{equation}
作用于所有名称中含“猛攻”的策略(全装猛攻、玻璃大炮猛攻、四套半装猛攻机密、全装单三猛攻),其他参数 $AR,AL,T$ 暂不调整,以保持模型简洁。

\subsubsection{情景 B:账号强度较低但爆率较高}

情景叙述:账号强度偏低,防御与操作略差,但地图爆率异常不错,使“绝密跑刀”类收益与成功率有所提升。

建模假设(相对提升):
\begin{equation}
AR' = 1.5 \times AR,\qquad
SR' = \min(1,\,1.1\times SR),
\end{equation}
作用于所有名称中含“绝密”且含“跑刀”的策略(绝密避战跑刀、经济学绝密跑刀),其他参数不变。

\subsection{情景 A:猛攻胜率提升 20\% 的影响}

对四种猛攻策略在 $\kappa=1$ 下重新计算 $RI^{*}$,得到:
\begin{itemize}
    \item 全装猛攻:$RI^{*}$ 由 0.010 提升至约 0.012;
    \item 玻璃大炮猛攻:由 0.021 提升至约 0.025;
    \item 四套半装猛攻机密:由 0.014 提升至约 0.017;
    \item 全装单三猛攻:由 0.018 提升至约 0.021。
\end{itemize}

其他非猛攻策略的 $RI^{*}$ 不变。情景 A 下的整体排名为:
\begin{enumerate}
    \item 普通经济学:0.640;
    \item 机密大坝速撤:0.245;
    \item 绝密避战跑刀:0.063;
    \item 经济学绝密跑刀:0.051;
    \item 玻璃大炮猛攻:0.025;
    \item 全装单三猛攻:0.021;
    \item 四套半装猛攻机密:0.017;
    \item 全装猛攻:0.012。
\end{enumerate}

猛攻类内部排序未变:玻璃大炮 $>$ 全装单三 $>$ 半装机密 $>$ 全装猛攻。与跑刀和速撤相比,差距略有缩小,但仍未进入 A 级区间。

\subsection{情景 B:绝密跑刀收益 +50\%,撤离率 +10\% 的影响}

在情景 B 下,两种绝密跑刀策略的 $RI^{*}$($\kappa=1$)变为:
\begin{itemize}
    \item 绝密避战跑刀:由 0.063 提升至约 0.104;
    \item 经济学绝密跑刀:由 0.051 提升至约 0.084。
\end{itemize}
其他策略保持不变。

情景 B 下整体排名为:
\begin{enumerate}
    \item 普通经济学:0.640;
    \item 机密大坝速撤:0.245;
    \item 绝密避战跑刀:0.104;
    \item 经济学绝密跑刀:0.084;
    \item 玻璃大炮猛攻:0.021;
    \item 全装单三猛攻:0.018;
    \item 四套半装猛攻机密:0.014;
    \item 全装猛攻:0.010。
\end{enumerate}

可以看到,两种绝密跑刀与大坝速撤之间的差距显著缩短,但仍未完全追平。

\subsection{排名变化的整体小结}

无论是情景 A 还是情景 B,名次顺序并未发生根本性变化:普通经济学与机密大坝速撤始终占据前两名;猛攻类始终处于末尾。情景 A 的主要作用是适度改善猛攻类经济性;情景 B 则显著抬升绝密跑刀的效率,使其从“高效”晋升为“接近顶级”的搬砖策略。

新增的全装单三猛攻在所有情景下均保持:经济性上略优于四套半装猛攻机密和全装猛攻,但仍明显不如玻璃大炮猛攻,更远不如任何跑刀或速撤策略。

\section{时间成本参数 $\kappa$ 的敏感性分析}

\subsection{一般形式}

从 $RI^{*}$ 的定义出发,对策略 $s$ 而言:
\begin{equation}
RI^{*}_s(\kappa) = \frac{\Delta R_{\text{adj},s}}{(C_s + \kappa T_s)\,T_s}
= \frac{\Delta R_{\text{adj},s}}{C_s T_s + \kappa T_s^2}.
\end{equation}

对固定策略,$\Delta R_{\text{adj},s},C_s,T_s$ 为常数,$\kappa$ 越大,分母越大,$RI^{*}_s(\kappa)$ 单调递减。长时间策略($T$ 大)随 $\kappa$ 增长下降更快。

\subsection{普通经济学与大坝速撤的交点}

以普通经济学与机密大坝速撤为例:

普通经济学:
\begin{equation}
\Delta R_{\text{adj}}^{(\text{普})} = 4,\quad
C^{(\text{普})} = 0,\quad
T^{(\text{普})} = 2.5,
\end{equation}
从而
\begin{equation}
RI^{*}_{\text{普}}(\kappa) = \frac{4}{(0 + 2.5\kappa)\times 2.5}
= \frac{4}{6.25\,\kappa} = \frac{0.64}{\kappa}.
\end{equation}

机密大坝速撤:
\begin{equation}
\Delta R_{\text{adj}}^{(\text{坝})} = 24.5,\quad
C^{(\text{坝})} = 15,\quad
T^{(\text{坝})} = 5,
\end{equation}
从而
\begin{equation}
RI^{*}_{\text{坝}}(\kappa) = \frac{24.5}{(15+5\kappa)\times 5}
= \frac{24.5}{75+25\kappa}.
\end{equation}

令二者相等:
\begin{equation}
\frac{0.64}{\kappa} = \frac{24.5}{75+25\kappa}.
\end{equation}
解得
\begin{equation}
0.64(75+25\kappa) = 24.5\kappa
\;\Rightarrow\;48 + 16\kappa = 24.5\kappa
\;\Rightarrow\; \kappa \approx 5.65.
\end{equation}

因此:
\begin{itemize}
    \item 当 $\kappa < 5.65$ 时,普通经济学的 $RI^{*}$ 略高于大坝速撤;
    \item 当 $\kappa > 5.65$ 时,大坝速撤反超,成为 $RI^{*}$ 第一名。
\end{itemize}

在本文采用的基准设定 $\kappa=1$ 下,普通经济学位列第一,大坝速撤位列第二,与上述分析一致。

\subsection{总体格局}

对 $\kappa\in[0.01,10]$ 进行数值扫描可以概括为:
\begin{itemize}
    \item 前两名始终由普通经济学与机密大坝速撤占据,仅在 $\kappa\approx 5.65$ 附近发生名次互换;
    \item 第三名始终在两种绝密跑刀之间切换,猛攻类策略未能在任何 $\kappa$ 取值下进入前三。
\end{itemize}

这说明:只要玩家对时间的价值不高到极端(例如 $\kappa$ 远大于 5--6 万/分钟),普通经济学与机密大坝速撤始终是经济效率上的主力,而跑刀策略是稳定的第二梯队;猛攻策略更多是“爽感取向”的选择,难以单凭经济指标上位。

\section{风险与爽度的扩展框架}

\subsection{风险与破产概率}

当前模型基于期望值,仅刻画平均收益,未显式刻画波动与爆仓风险。后续可以引入:
\begin{itemize}
    \item 收益分布、方差、VaR(风险价值)、CVaR(条件风险价值)等风险指标;
    \item 有限初始资金 $W_0$,在给定策略下,分析连续 $N$ 局后的破产概率;
    \item 在指定破产概率约束(如 5\%)下的最优仓位与策略组合。
\end{itemize}

这有助于回答“小仓是否适合极端猛攻”“长期使用某策略是否容易爆仓”等问题。

\subsection{爽度--经济效率双目标}

不同玩家的目标往往并非只赚钱,而是在赚钱与爽感之间权衡。可以引入主观爽度评分 $S_s\in[1,10]$(或拆分为爽感与练度收益),则每个策略在平面上对应一点:
\begin{equation}
(RI^{*}_s,\;S_s).
\end{equation}

绘制散点图可得到:
\begin{itemize}
    \item 右上角:高效率 + 高爽度(若存在,则为真正意义上的“神仙策略”);
    \item 右下角:高效率、低爽度(典型如普通经济学、大坝速撤,适合“搬砖日”);
    \item 左上角:低效率、高爽度(各类极端猛攻,适合“今晚只想爽”的场景);
    \item 左下角:低效率、低爽度(在实践中应尽量避免)。
\end{itemize}

该双目标框架有助于向不同风格玩家推荐不同策略,并用“效率--爽度前沿”定义真正意义上的理性爽玩法。

\section{结论与展望}

\subsection{主要结论}

\begin{enumerate}
    \item 在基准时间成本 $\kappa=1$ 下,普通经济学与机密大坝速撤在 $RI^{*}$ 视角下位列前两名(0.640 与 0.245),构成经济学意义上的 S 级效率组合。
    \item 在原始参数下,绝密避战跑刀与经济学绝密跑刀位列 A 级(0.063 与 0.051),是仅次于 S 级策略的高效搬砖方案;在“低号高爆率”情景 B 中,其 $RI^{*}$ 分别提升至 0.104 与 0.084,进一步逼近 S 级水平。
    \item 各类猛攻策略在 $RI^{*}$ 视角下均处于 B 级区间,即使在固排与技术提升(情景 A)下,其经济效率仍明显落后于跑刀与速撤:玻璃大炮猛攻与全装单三猛攻可视为猛攻中的“相对理性”选择;四套半装猛攻机密与全装猛攻更接近高爽度课金玩法。
    \item 时间成本单价 $\kappa$ 的敏感性分析表明:在较宽的 $\kappa$ 区间内,前两名始终由普通经济学与机密大坝速撤占据,仅在 $\kappa\approx 5.65$ 附近发生名次互换;第三名始终由两种绝密跑刀占据,猛攻类无法进入前三。
    \item 辅助指标 $J_1,J_2$ 提供了有意义的补充视角:$J_1$ 体现单局爆发与每分钟赚钱速度,使全装单三猛攻、绝密避战跑刀的爽感优势更加显性;$J_2$ 则直观展示了普通经济学与大坝速撤在资本效率上的优势。
\end{enumerate}

\subsection{模型局限与后续工作}

当前模型仍存在若干局限与可拓展方向:
\begin{itemize}
    \item 参数取值基于经验设定,真实环境下的 $AR,AL,SR,T$ 具有波动性与样本偏差,未来可基于真实数据进行估计,并给出 $RI^{*}$ 的置信区间与敏感度分析;
    \item 模型主要刻画的是金钱与时间两个维度,尚未显式刻画爽度、练度提升等非金钱效用;通过引入爽度评分,可以构建“经济效率--爽度”双目标决策框架;
    \item 目前仅使用期望值指标,未引入收益分布与破产概率,对小仓玩家尤其不够安全;后续可以引入 VaR/CVaR、马尔可夫链或蒙特卡洛模拟,对不同策略组合下的长期生存率进行系统分析;
    \item 本文将策略视为离散选项,更进一步可以将成本与打法强度视为连续可调决策变量,在预算与时间约束下求解最优进攻性水平。
\end{itemize}

总体而言,本文给出了一套在游戏/战术场景下可落地的经济学评价框架:以三角洲收益指数 $RI^{*}$ 为核心,辅以 $J_1,J_2$ 与情景化调整,可以为不同风格、不同资源约束的玩家或队伍提供清晰的策略选择依据。未来在引入真实数据与风险--爽度扩展之后,该框架有望进一步发展为一套完整的“战术金融工程工具箱”。

\end{document}
```
